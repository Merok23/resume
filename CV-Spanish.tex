%% The MIT License (MIT)
%%
%% Copyright (c) 2015 Daniil Belyakov
%%
%% Permission is hereby granted, free of charge, to any person obtaining a copy
%% of this software and associated documentation files (the "Software"), to deal
%% in the Software without restriction, including without limitation the rights
%% to use, copy, modify, merge, publish, distribute, sublicense, and/or sell
%% copies of the Software, and to permit persons to whom the Software is
%% furnished to do so, subject to the following conditions:
%%
%% The above copyright notice and this permission notice shall be included in all
%% copies or substantial portions of the Software.
%%
%% THE SOFTWARE IS PROVIDED "AS IS", WITHOUT WARRANTY OF ANY KIND, EXPRESS OR
%% IMPLIED, INCLUDING BUT NOT LIMITED TO THE WARRANTIES OF MERCHANTABILITY,
%% FITNESS FOR A PARTICULAR PURPOSE AND NONINFRINGEMENT. IN NO EVENT SHALL THE
%% AUTHORS OR COPYRIGHT HOLDERS BE LIABLE FOR ANY CLAIM, DAMAGES OR OTHER
%% LIABILITY, WHETHER IN AN ACTION OF CONTRACT, TORT OR OTHERWISE, ARISING FROM,
%% OUT OF OR IN CONNECTION WITH THE SOFTWARE OR THE USE OR OTHER DEALINGS IN THE
%% SOFTWARE.

% The font could be set to Windows-specific Calibri by using the 'calibri' option
\documentclass[]{mcdowellcv}

% For mathematical symbols
\usepackage{amsmath}

% Set applicant's personal data for header
\name{Caliz Blanco,\linebreak Alejo Martin Ezequiel}
\address{2922 Domingo de Acassuso \linebreak Olivos Buenos Aires 1636}
\contacts{(+54) 011 32935066 \linebreak alejocalizblanco@gmail.com}

\begin{document}

	% Imprimir el encabezado
	\makeheader
	
	% Imprimir el contenido
	\begin{cvsection}{Experiencia Laboral}
		\begin{cvsubsection}{Desarrollador Full Stack}{NDA}{Enero 2024 -- Presente}
			NDA App
			\begin{itemize}
				\item Desarrollé diversas funcionalidades tanto para nutricionistas como para pacientes.
				\item Actualicé parte de la infraestructura reemplazando tecnologías obsoletas por soluciones estándar de la industria.
				\item Implementé un modelo de negocio multi-tenant reutilizando el código existente de la aplicación.
			\end{itemize}
		\end{cvsubsection}
		
		\begin{cvsubsection}{Ayudante de Primera}{Facultad de Ingeniería, Universidad de Buenos Aires}{Agosto 2022 -- Presente}
			Ciencia de Datos
			\begin{itemize}
				\item Diseñé y evalué diversos trabajos prácticos enfocados en análisis de datos, manipulación de datos y aprendizaje automático.
				\item Preparé y dicté clases para grupos de más de 100 estudiantes.
			\end{itemize}
		\end{cvsubsection}
		
		\begin{cvsubsection}{Ayudante de Primera}{Facultad de Ingeniería, Universidad de Buenos Aires}{Febrero 2024 -- Presente}
			Sistemas de Bases de Datos
			\begin{itemize}
				\item Diseñé y evalué exámenes sobre temas como SQL, NoSQL, concurrencia en bases de datos, bases de datos distribuidas, diagramas entidad-relación, álgebra relacional, entre otros.
				\item Preparé y dicté clases para grupos de más de 100 estudiantes.
			\end{itemize}
		\end{cvsubsection}
	\end{cvsection}
	
	\begin{cvsection}{Educación}
		\begin{cvsubsection}{CABA, Buenos Aires}{Facultad de Ingeniería, Universidad de Buenos Aires}{Julio 2018 -- Febrero 2025}
			Ingeniería en Software
			\begin{itemize}
				\item Arquitectura de software, programación concurrente y sistemas distribuidos.
				\item Sistemas operativos, bases de datos y estructuras de datos.
				\item Ciencia de datos, análisis de datos e ingeniería de datos.
				\item Algoritmos, arquitectura de computadoras y principios de ingeniería de software.
				\item Metodologías de programación y buenas prácticas.
				\item Física, matemáticas y estadística.
			\end{itemize}
		\end{cvsubsection}
	\end{cvsection}	
	
	\begin{cvsection}{Experiencia Técnica}
		\begin{cvsubsection}{Proyectos}{}{}
			\begin{itemize}
				\item \textbf{Ahorratón} (2024). Una PWA diseñada para ayudar a los usuarios a ahorrar dinero comparando precios en diferentes supermercados. Tecnologías utilizadas: Python, React, MUI, SQL y Docker.
				\item \textbf{Aplicación Social Similar a Twitter} (2023). Una aplicación de red social basada en microservicios que permite a los usuarios publicar "tweets," subir imágenes, seguir a otros usuarios y más. Tecnologías utilizadas: Python, Go, JavaScript, React, SQL, NoSQL, Docker y Kubernetes.
				\item \textbf{Clon 2D de Left 4 Dead} (2023). Un juego multijugador cooperativo en línea que soporta múltiples sesiones simultáneas, construido con una arquitectura cliente-servidor. Implementado en C++.
			\end{itemize}
		\end{cvsubsection}
	\end{cvsection}
	
	\begin{cvsection}{Lenguajes y Tecnologías}
		\begin{cvsubsection}{}{}{}
			\begin{itemize}
				\item \textbf{Proficiente en:}
				\subitem Python; SQL; React; Docker; Git; UNIX; NoSQL; Pandas; Spark; Selenium (Web Scraping)
				\item \textbf{Competente en:}
				\subitem Go (Golang); C++; JavaScript; Vue
				\item \textbf{Familiarizado y dispuesto a aprender:}
				\subitem Rust; C; Ruby; Angular; Flutter; Swift; AWS; Azure; GCP
			\end{itemize}
		\end{cvsubsection}
	\end{cvsection}
	
	
\end{document}
